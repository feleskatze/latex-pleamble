
%オクマクロ
%傍点=\kenten{}
%ルビ=\ruby{漢字}{かんじ}
\usepackage{okumacro}

%記号用
\usepackage{amssymb}
%レイアウト確認用パッケージ
\usepackage{layout}
%部分的段組 begin{multicols}{2}
\usepackage{multicol}

%レイアウト・余白指定
\usepackage[dvipdfm,truedimen]{geometry}
\geometry{left=20truemm,right=20truemm,top=20truemm,bottom=20truemm}


%枠など
\usepackage{tcolorbox}
\usepackage{ascmac}

% 図
% \usepackage[dviout]{graphicx}

% 図の回り込み
\usepackage{wrapfig}

% URL
\usepackage{url}

% % ハイパーリンク
\usepackage[dvipdfmx,bookmarks=true,bookmarksnumbered=true]{hyperref}
\usepackage{pxjahyper}

%ツリー図
%\usepackage{tikz,tikz-qtree}

%==================
%行数・文字数は行間・文字間隔を調整するのが一番無難っぽい
%==================
%行数表示
%\pagewiselinenumbers
%行間
\renewcommand{\baselinestretch}{0.9}
%文字間隔(begin{document}の後に書いて)
\kanjiskip=1.5zw plus 3pt minus 3pt
\xkanjiskip=1.5zw plus 3pt minus 3pt


%下使えない(効果なし)
%文字数・行数指定用マクロ(http://d.hatena.ne.jp/Rion778/20091002/1254482262)
%上使えないがURLは残しておく

% フォントセットの変更(Windowsの場合はコメントアウト)
% \usepackage[deluxe]{otf}
%ウムラウトなどを表示するため
\usepackage[utf8]{inputenc}
\usepackage[T1]{fontenc}
\usepackage{lmodern}

%ちゃんとインデントされるように
\usepackage{indentfirst}

\pagestyle{headings}

%タイトル
\makeatletter
\def\thickhrulefill{\leavevmode \leaders \hrule height 1pt\hfill \kern \z@}
\renewcommand{\maketitle}{\begin{titlepage}%
    \let\footnotesize\small
    \let\footnoterule\relax
    \parindent \z@
    \reset@font
    \null
    \vskip 10\p@
    \hbox{\mbox{\hspace{3em}}%
      \vrule depth 0.6\textheight%
      \mbox{\hspace{2em}}
      \vbox{
        \vskip 130\p@
        \begin{flushleft}
          \HUGE \bfseries \@title \par
        \end{flushleft}
        \vfil
        \vskip 60\p@
        {\Large\hfil\normalfont \@date}%
        }}
    \null
  \end{titlepage}%
  \setcounter{footnote}{0}%
}
\makeatother

%目次関係
\renewcommand{\contentsname}{\mcfamily 目次}
\renewcommand{\headfont}{\mcfamily \rmfamily}
\setcounter{tocdepth}{2}

%====================
%セクション体裁変更
%====================
\setcounter{secnumdepth}{3}
\usepackage{titlesec}
\usepackage{picture}

%chapter
\titleformat{\chapter}[block]
{}{}{0pt}{
	\fontsize{30pt}{30pt}\selectfont\filleft\hrule
}[
	\hrule \smallskip \hrule \Large{\filleft 第 \thechapter 章}
]
\titlespacing{\chapter}{0pt}{*2}{10truemm}

%section
\titleformat{\section}[block]
{}{}{0pt}{
	\normalfont \Large\thesection
}[\hrule]
\titlespacing{\section}{0pt}{2zh}{1zh}

%subsection
\titlespacing{\subsection}{0.5zw}{1zh}{0zh}

%subsubsection
\titlespacing{\subsubsection}{0.7zw}{1zh}{0zh}

%paragraphの変更(前後改段落・明朝体)
\renewcommand{\paragraph}[1]{\par \medskip \mcfamily\underline{□#1} \medskip \par}
\titlespacing{\paragraph}{0.5zw}{2zh}{1zh}
