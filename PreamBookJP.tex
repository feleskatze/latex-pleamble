% \documentclass[uplatex, tate, hanging_punctuation, paper=b6, reference_mark=interlinear, book]{jlreq}

% フォントセットの変更(Windowsの場合はコメントアウト)
% \usepackage[deluxe]{otf}
%ウムラウトなどを表示するため
\usepackage[utf8]{inputenc}
\usepackage[T1]{fontenc}
\usepackage{lmodern}
%% 図表など
% 図の読みこみのために
\usepackage[dvipdfmx, hiresbb]{graphicx, xcolor}
\usepackage{booktabs} % 表の横罫線

%% 囲み枠
\usepackage{tcolorbox}
\tcbuselibrary{breakable} % ページをまたいで分割できるように

%% misc
\usepackage{okumacro} % 圏点などのために
\usepackage{pxrubrica} % ルビをつける(okumacroのrubyは使わない

%% 見出しのスタイルの設定
% chapterの定義
\DeclareTobiraHeading{chapter}{1}{% chapter を扉見出しに
  format={\null\vfil {\huge\gtfamily\bfseries {\LARGE #1}#2}}, % 見出しのフォント
  label_format={第\thechapter 章\hspace{2zw}} % ラベルのフォーマット
}
\DeclareBlockHeading{chapter}{1}{ % chapter を別行見出しに
  pagebreak=cleardoublepage, % 章を始める前に改丁
  label_format={第\thechapter 章}, % ラベルのフォーマット
  font={\gtfamily\LARGE}, % 見出しのフォント
  lines=3,after_lines=2, % 見出しのために5行取り、後ろの方が2行分広い
  indent=2zw % インデント
}

% section の定義
\renewcommand{\thesection}{} % 節の番号はなしが基本
\DeclareBlockHeading{section}{2}{ % section を別行見出しに
  font={\gtfamily}, % 見出しのフォント
  lines=1, before_lines=1% 見出しの前に1行取る
}

%% 目次の設定
\setcounter{tocdepth}{1} % sectionまでを目次に

%% hyperrefの設定
\usepackage[dvipdfmx,%
    pdftitle=タイトル, % PDFのタイトル
    pdfauthor=作成者, % PDFの作成者
    bookmarks=true, % PDFにしおりをつける
    bookmarksnumbered=true, % しおりに節番号などをつける
    colorlinks=false, % リンクには色をつけない
    hyperfootnotes=false, % 脚注からのリンクを作らない
    pdfborder={0 0 0}, % 枠なし
    pdfdirection=R2L, % 開く方向
    pdfpagelayout=TwoPageRight, % 奇数頁が右側になるような見開きモードで開く
    pdfpagemode=UseNone]{hyperref}

% PDFにしたときのしおりの文字化けを防ぐ
\usepackage{pxjahyper}

\usepackage{pxeveryshi}
% hyperref を使っているときに
% 目次でのページ番号の向きを適切にする
\makeatletter
\def\contentsline#1#2#3#4{\csname l@#1\endcsname{\hyper@linkstart{link}{#4}{#2}\hyper@linkend}{\rensuji{#3}}}
\makeatother
